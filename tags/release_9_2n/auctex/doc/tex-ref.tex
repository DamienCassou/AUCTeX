+% -*- plain-tex -*- 
% $Id: tex-ref.tex,v 1.3 1994-05-04 22:54:43 amanda Exp $
% Reference Card for Auc TeX version 7
%**start of header
\newcount\columnsperpage

% This file can be printed with 1, 2, or 3 columns per page (see below).
% Specify how many you want here.  Nothing else needs to be changed.

\columnsperpage=2

% This file is intended to be processed by plain TeX (TeX82).
% compile-command: "tex ref-card.tex"
%
% Original author of Auc-TeX Reference Card:
%                                  
%       Terrence Brannon, PO Box 5027, Bethlehem, PA 18015 , USA
%  internet: tb06@pl118f.cc.lehigh.edu  (215) 758-1720 (215) 758-2104
%
% Kresten Krab Thorup updated the reference card to 6.
% Per Abrahamsen updated the reference card to 7, 8, and 9.
%
% Thanks to Stephen Gildea
% Paul Rubin, Bob Chassell, Len Tower, and Richard Mlynarik
% for creating the GNU Emacs Reference Card from which this was mutated

\def\versionnumber{9}
\def\year{1993}
\def\version{February \year\ v\versionnumber}

\def\shortcopyrightnotice{\vskip 1ex plus 2 fill
  \centerline{\small \copyright\ \year\ Free Software Foundation, Inc.
  Permissions on back.  v\versionnumber}}

\def\copyrightnotice{
\vskip 1ex plus 2 fill\begingroup\small
\centerline{Copyright \copyright\ 1987 Free Software Foundation, Inc.}
\centerline{Copyright \copyright\ 1992 Kresten Krab Thorup}
\centerline{Copyright \copyright\ \year\ Per Abrahamsen}
\centerline{for AUC \TeX\ version \versionnumber}

Permission is granted to make and distribute copies of
this card provided the copyright notice and this permission notice
are preserved on all copies.


\endgroup}

% make \bye not \outer so that the \def\bye in the \else clause below
% can be scanned without complaint.
\def\bye{\par\vfill\supereject\end}

\newdimen\intercolumnskip
\newbox\columna
\newbox\columnb

\def\ncolumns{\the\columnsperpage}

\message{[\ncolumns\space 
  column\if 1\ncolumns\else s\fi\space per page]}

\def\scaledmag#1{ scaled \magstep #1}

% This multi-way format was designed by Stephen Gildea
% October 1986.
\if 1\ncolumns
  \hsize 4in
  \vsize 10in
  \voffset -.7in
  \font\titlefont=\fontname\tenbf \scaledmag3
  \font\headingfont=\fontname\tenbf \scaledmag2
  \font\smallfont=\fontname\sevenrm
  \font\smallsy=\fontname\sevensy

  \footline{\hss\folio}
  \def\makefootline{\baselineskip10pt\hsize6.5in\line{\the\footline}}
\else
  \hsize 3.2in
  \vsize 7.95in
%  \hoffset -.75in
%  \voffset -.745in
  \font\titlefont=cmbx10 \scaledmag2
  \font\headingfont=cmbx10 \scaledmag1
  \font\smallfont=cmr6
  \font\smallsy=cmsy6
  \font\eightrm=cmr8
  \font\eightbf=cmbx8
  \font\eightit=cmti8
  \font\eighttt=cmtt8
  \font\eightsy=cmsy8
  \textfont0=\eightrm
  \textfont2=\eightsy
  \def\rm{\eightrm}
  \def\bf{\eightbf}
  \def\it{\eightit}
  \def\tt{\eighttt}
  \normalbaselineskip=.8\normalbaselineskip
  \normallineskip=.8\normallineskip
  \normallineskiplimit=.8\normallineskiplimit
  \normalbaselines\rm           %make definitions take effect

  \if 2\ncolumns
    \let\maxcolumn=b
    \footline{\hss\rm\folio\hss}
    \def\makefootline{\vskip 2in \hsize=6.86in\line{\the\footline}}
  \else \if 3\ncolumns
    \let\maxcolumn=c
    \nopagenumbers
  \else
    \errhelp{You must set \columnsperpage equal to 1, 2, or 3.}
    \errmessage{Illegal number of columns per page}
  \fi\fi

  \intercolumnskip=.46in
  \def\abc{a}
  \output={%
      % This next line is useful when designing the layout.
      %\immediate\write16{Column \folio\abc\space starts with \firstmark}
      \if \maxcolumn\abc \multicolumnformat \global\def\abc{a}
      \else\if a\abc
        \global\setbox\columna\columnbox \global\def\abc{b}
        %% in case we never use \columnb (two-column mode)
        \global\setbox\columnb\hbox to -\intercolumnskip{}
      \else
        \global\setbox\columnb\columnbox \global\def\abc{c}\fi\fi}
  \def\multicolumnformat{\shipout\vbox{\makeheadline
      \hbox{\box\columna\hskip\intercolumnskip
        \box\columnb\hskip\intercolumnskip\columnbox}
      \makefootline}\advancepageno}
  \def\columnbox{\leftline{\pagebody}}

  \def\bye{\par\vfill\supereject
    \if a\abc \else\null\vfill\eject\fi
    \if a\abc \else\null\vfill\eject\fi
    \end}  
\fi

% we won't be using math mode much, so redefine some of the characters
% we might want to talk about
\catcode`\^=12
\catcode`\_=12

\chardef\\=`\\
\chardef\{=`\{
\chardef\}=`\}

\hyphenation{mini-buf-fer}

\parindent 0pt
\parskip 1ex plus .5ex minus .5ex

\def\small{\smallfont\textfont2=\smallsy\baselineskip=.8\baselineskip}

\outer\def\newcolumn{\vfill\eject}

\outer\def\title#1{{\titlefont\centerline{#1}}\vskip 1ex plus .5ex}

\outer\def\section#1{\par\filbreak
  \vskip 3ex plus 2ex minus 2ex {\headingfont #1}\mark{#1}%
  \vskip 2ex plus 1ex minus 1.5ex}

\newdimen\keyindent

\def\beginindentedkeys{\keyindent=1em}
\def\endindentedkeys{\keyindent=0em}
\endindentedkeys

\def\paralign{\vskip\parskip\halign}

\def\<#1>{$\langle${\rm #1}$\rangle$}

\def\kbd#1{{\tt#1}\null}        %\null so not an abbrev even if period follows

\def\beginexample{\par\leavevmode\begingroup
  \obeylines\obeyspaces\parskip0pt\tt}
{\obeyspaces\global\let =\ }
\def\endexample{\endgroup}

\def\key#1#2{\leavevmode\hbox to \hsize{\vtop
  {\hsize=.75\hsize\rightskip=1em
  \hskip\keyindent\relax#1}\kbd{#2}\hfil}}

\newbox\metaxbox
\setbox\metaxbox\hbox{\kbd{M-x }}
\newdimen\metaxwidth
\metaxwidth=\wd\metaxbox

\def\metax#1#2{\leavevmode\hbox to \hsize{\hbox to .75\hsize
  {\hskip\keyindent\relax#1\hfil}%
  \hskip -\metaxwidth minus 1fil
  \kbd{#2}\hfil}}

\def\threecol#1#2#3{\hskip\keyindent\relax#1\hfil&\kbd{#2}\quad
  &\kbd{#3}\quad\cr}

%**end of header


\title{AUC \TeX\ Reference Card}

\centerline{(for version \versionnumber)}

\section{Conventions Used}

\key{Carriage Return}{RET}
\key{Tabular}{TAB}
\key{Linefeed}{LFD}

Mode variables: You want to change the variables found in the file
`{\tt tex-site.el}' in the AUC-\TeX\ distribution for your site.  The
other variables should be set by the individual user.  

Entering LaTeX mode calls the value of text-mode-hook, then the value
of TeX-mode-hook, and then the value of LaTeX-mode-hook.

\section{Shell Interaction}

\key{Save Document}{C-c C-d}
\key{Run a command on the master file}{C-c C-c}
\key{Run a command on the buffer}{C-c C-b}
\key{Run a command on the region}{C-c C-r}
\key{Kill job}{C-c C-k}
\key{Recenter output buffer}{C-c C-l}
\key{Next error in \TeX/LaTeX session}{C-c `}
\key{Toggle debug of wonderful boxes}{C-c C-w}
\key{Switch to master file or active buffer}{C-c ^}

Commands you can run on the master file (with C-c C-c) or the region
(with C-c C-r) include the following.

\key{\TeX}{TeX}
\overfullrule=0pt %The next line is too wide.
\key{Run \TeX{} Interactively}{TeX Interactive}
\key{LaTeX}{LaTeX}
\key{Run LaTeX Interactively}{LaTeX Interactive}
\key{SliTeX}{SliTeX}
\key{A previewer}{View}
\key{Printing the DVI file}{Print}
\key{Bib\TeX}{BibTeX}
\key{MakeIndex}{Index}
\key{LaCheck}{Check}
\key{(PostScript) File}{File}
\key{Ispell}{Spell} 

\section{Command Insertion}

\key{Insert Section}{C-c C-s}
\key{Insert LaTeX environment}{C-c C-e}
\key{Insert item}{C-c LFD}
\key{Close LaTeX environment}{C-c ]}
\key{Insert \TeX\ macro \kbd{\\\{\}} }{C-c C-m}
\key{Insert double brace}{C-c \{}
\key{Complete \TeX\ macro}{M-TAB}
\key{Smart ``quote''}{"}
\key{Smart ``dollar''}{\$}
  
\section{Font Selection}

\key{Insert {\bf bold} text}{C-c C-f C-b}
\key{Insert {\it italics\/} text}{C-c C-f C-i}
\key{Insert {\rm roman} text}{C-c C-f C-r}
\key{Insert {\it emphasized} text}{C-c C-f C-e}
\key{Insert {\tt typewriter} text}{C-c C-f C-t}
\key{Insert {\sl slanted\/} text}{C-c C-f C-s}
\key{Delete font}{C-c C-f C-d}
\key{Change font}{C-u C-c C-f <KEY>}

\section{Source Formatting}

\key{Indent current line}{TAB}
\key{Indent next line}{LFD}

\key{Format a paragraph}{M-q}
\key{Format a region}{C-c C-q C-r}
\key{Format a section}{C-c C-q C-s}
\key{Format an environment}{C-c C-q C-e}

\key{Mark an environment}{C-c .}
\key{Mark a section}{C-c *}

\key{Comment region}{C-c ;}
\key{Comment paragraph}{C-c \%}
\key{Uncomment region}{C-u - C-c ;}
\key{Uncomment paragraph}{C-u - C-c \%}

\section{Miscellaneous}

\key{Math Mode}{C-c \~{}}
\key{Reset AUC TeX}{C-c C-n}

\copyrightnotice

\bye

% End:
