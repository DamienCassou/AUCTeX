% -*- plain-tex -*- 
% $Id: math-ref.tex,v 1.7 1994-05-28 02:47:56 amanda Exp $
% Reference Card for LaTeX Math Minor Mode
%**start of header
\newcount\columnsperpage

% This file can be printed with 1, 2, or 3 columns per page (see below).
% Specify how many you want here.  Nothing else needs to be changed.

\columnsperpage=2

% This file is intended to be processed by plain TeX (TeX82).
% compile-command: "tex auc-tex-ref.tex"

% Original author of Auc-TeX Reference Card:
%                                  
%       Terrence Brannon, PO Box 5027, Bethlehem, PA 18015 , USA
%  internet: tb06@pl118f.cc.lehigh.edu  (215) 758-1720 (215) 758-2104
%
% Kresten Krab Thorup updated the reference card to 6.*
% Per Abrahamsen updated the reference card to 7.*
%
% Thanks to Stephen Gildea
% Paul Rubin, Bob Chassell, Len Tower, and Richard Mlynarik
% for creating the GNU Emacs Reference Card from which this was mutated

\def\versionnumber{5.2}
\def\year{1993}
\def\version{February \year\ v\versionnumber}

\def\shortcopyrightnotice{\vskip 1ex plus 2 fill
  \centerline{\small \copyright\ \year\ Free Software Foundation, Inc.
  Permissions on back.  v\versionnumber}}

\def\copyrightnotice{
\vskip 1ex plus 2 fill\begingroup\small
\centerline{Copyright \copyright\ 1987 Free Software Foundation, Inc.}
\centerline{Copyright \copyright\ 1992 Kresten Krab Thorup}
\centerline{Copyright \copyright\ \year\ Per Abrahamsen}

Permission is granted to make and distribute copies of
this card provided the copyright notice and this permission notice
are preserved on all copies.

\endgroup}

% make \bye not \outer so that the \def\bye in the \else clause below
% can be scanned without complaint.
\def\bye{\par\vfill\supereject\end}

\newdimen\intercolumnskip
\newbox\columna
\newbox\columnb

\def\ncolumns{\the\columnsperpage}

\message{[\ncolumns\space 
  column\if 1\ncolumns\else s\fi\space per page]}

\def\scaledmag#1{ scaled \magstep #1}

% This multi-way format was designed by Stephen Gildea
% October 1986.
\if 1\ncolumns
  \hsize 4in
  \vsize 10in
  \voffset -.7in
  \font\titlefont=\fontname\tenbf \scaledmag3
  \font\headingfont=\fontname\tenbf \scaledmag2
  \font\smallfont=\fontname\sevenrm
  \font\smallsy=\fontname\sevensy

  \footline{\hss\folio}
  \def\makefootline{\baselineskip10pt\hsize6.5in\line{\the\footline}}
\else
  \hsize 3.2in
  \vsize 7.95in
%  \hoffset -.75in
%  \voffset -.745in
  \font\titlefont=cmbx10 \scaledmag2
  \font\headingfont=cmbx10 \scaledmag1
  \font\smallfont=cmr6
  \font\smallsy=cmsy6
  \font\eightrm=cmr8
  \font\eightbf=cmbx8
  \font\eightit=cmti8
  \font\eighttt=cmtt8
  \font\eightsy=cmsy8
  \textfont0=\eightrm
  \textfont2=\eightsy
  \def\rm{\eightrm}
  \def\bf{\eightbf}
  \def\it{\eightit}
  \def\tt{\eighttt}
  \normalbaselineskip=.8\normalbaselineskip
  \normallineskip=.8\normallineskip
  \normallineskiplimit=.8\normallineskiplimit
  \normalbaselines\rm           %make definitions take effect

  \if 2\ncolumns
    \let\maxcolumn=b
    \footline{\hss\rm\folio\hss}
    \def\makefootline{\vskip 2in \hsize=6.86in\line{\the\footline}}
  \else \if 3\ncolumns
    \let\maxcolumn=c
    \nopagenumbers
  \else
    \errhelp{You must set \columnsperpage equal to 1, 2, or 3.}
    \errmessage{Illegal number of columns per page}
  \fi\fi

  \intercolumnskip=.46in
  \def\abc{a}
  \output={%
      % This next line is useful when designing the layout.
      %\immediate\write16{Column \folio\abc\space starts with \firstmark}
      \if \maxcolumn\abc \multicolumnformat \global\def\abc{a}
      \else\if a\abc
        \global\setbox\columna\columnbox \global\def\abc{b}
        %% in case we never use \columnb (two-column mode)
        \global\setbox\columnb\hbox to -\intercolumnskip{}
      \else
        \global\setbox\columnb\columnbox \global\def\abc{c}\fi\fi}
  \def\multicolumnformat{\shipout\vbox{\makeheadline
      \hbox{\box\columna\hskip\intercolumnskip
        \box\columnb\hskip\intercolumnskip\columnbox}
      \makefootline}\advancepageno}
  \def\columnbox{\leftline{\pagebody}}

  \def\bye{\par\vfill\supereject
    \if a\abc \else\null\vfill\eject\fi
    \if a\abc \else\null\vfill\eject\fi
    \end}  
\fi

% we won't be using math mode much, so redefine some of the characters
% we might want to talk about
\catcode`\^=12
\catcode`\_=12

\chardef\\=`\\
\chardef\{=`\{
\chardef\}=`\}

\hyphenation{mini-buf-fer}

\parindent 0pt
\parskip 1ex plus .5ex minus .5ex

\def\small{\smallfont\textfont2=\smallsy\baselineskip=.8\baselineskip}

\outer\def\newcolumn{\vfill\eject}

\outer\def\title#1{{\titlefont\centerline{#1}}\vskip 1ex plus .5ex}

\outer\def\section#1{\par\filbreak
  \vskip 3ex plus 2ex minus 2ex {\headingfont #1}\mark{#1}%
  \vskip 2ex plus 1ex minus 1.5ex}

\newdimen\keyindent

\def\beginindentedkeys{\keyindent=1em}
\def\endindentedkeys{\keyindent=0em}
\endindentedkeys

\def\paralign{\vskip\parskip\halign}

\def\<#1>{$\langle${\rm #1}$\rangle$}

\def\kbd#1{{\tt#1}\null}        %\null so not an abbrev even if period follows

\def\beginexample{\par\leavevmode\begingroup
  \obeylines\obeyspaces\parskip0pt\tt}
{\obeyspaces\global\let =\ }
\def\endexample{\endgroup}

\def\key#1#2{\leavevmode\hbox to \hsize{\vtop
  {\hsize=.75\hsize\rightskip=1em
  \hskip\keyindent\relax#1}\kbd{#2}\hfil}}

\newbox\metaxbox
\setbox\metaxbox\hbox{\kbd{M-x }}
\newdimen\metaxwidth
\metaxwidth=\wd\metaxbox

\def\metax#1#2{\leavevmode\hbox to \hsize{\hbox to .75\hsize
  {\hskip\keyindent\relax#1\hfil}%
  \hskip -\metaxwidth minus 1fil
  \kbd{#2}\hfil}}

\def\threecol#1#2#3{\hskip\keyindent\relax#1\hfil&\kbd{#2}\quad
  &\kbd{#3}\quad\cr}

%**end of header


\title{Math Mode Reference Card}

\section{Variables}

All math mode commands are under the prefix key specified by
LaTeX-math-abbrev-prefix, default is "`".

You can define your own math mode commands by setting the variable
LaTeX-math-list before loading LaTeX-math-mode.

\section{Greek Letters}

\def\disp#1{\hbox to 6ex{$#1$\hfill}}
\key{\disp{\alpha} (alpha)}{a}
\key{\disp{\beta} (beta)}{b}
\key{\disp{\delta} (delta)}{d}
\key{\disp{\epsilon} (epsilon)}{e}
\key{\disp{\phi} (phi)}{f}
\key{\disp{\gamma} (gamma)}{g}
\key{\disp{\eta} (eta)}{h}
\key{\disp{\kappa} (kappa)}{k}
\key{\disp{\lambda} (lambda)}{l}
\key{\disp{\mu} (mu)}{m}
\key{\disp{\nabla} (nabla)}{N}
\key{\disp{\nu} (nu)}{n}
\key{\disp{\omega} (omega)}{o}
\key{\disp{\pi} (pi)}{p}
\key{\disp{\theta} (theta)}{q}
\key{\disp{\rho} (rho)}{r}
\key{\disp{\sigma} (sigma)}{s}
\key{\disp{\tau} (tau)}{t}
\key{\disp{\upsilon} (upsilon)}{u}
\key{\disp{\chi} (chi)}{x}
\key{\disp{\psi} (psi)}{y}
\key{\disp{\zeta} (zeta)}{z}
\key{\disp{\Delta} (Delta)}{D}
\key{\disp{\Gamma} (Gamma)}{G}
\key{\disp{\Theta} (Theta)}{Q}
\key{\disp{\Lambda} (Lambda)}{L}
\key{\disp{\Psi} (Psi)}{Y}
\key{\disp{\Pi} (Pi)}{P}
\key{\disp{\Sigma} (Sigma)}{S}
\key{\disp{\Upsilon} (Upsilon)}{U}
\key{\disp{\Phi} (Phi)}{V}
\key{\disp{\Omega} (Omega)}{O}

\section{Symbols}

\key{\disp{\rightarrow} (rightarrow)}{C-f}
\key{\disp{\leftarrow} (leftarrow)}{C-b}
\key{\disp{\uparrow} (uparrow)}{C-p}
\key{\disp{\downarrow} (downarrow)}{C-n}
\key{\disp{\leq} (leq)}{<}
\key{\disp{\geq} (geq)}{>}
\key{\disp{\tilde{ }} (tilde)}{$\tilde{ }$}
\key{\disp{\infty} (infty)}{I}
\key{\disp{\forall} (forall)}{A}
\key{\disp{\exists} (exists)}{E}
\key{\disp{\not } (not)}{!}
\key{\disp{\in} (in)}{i}
\key{\disp{\times} (times)}{*}
\key{\disp{\cdot} (cdot)}{.}
\key{\disp{\subset} (subset)}{\{}
\key{\disp{\supset} (supset)}{\}}
\key{\disp{\subseteq} (subseteq)}{[}
\key{\disp{\supseteq} (supseteq)}{]}
\key{\disp{\backslash} (backslash)}{\\}
\key{\disp{\setminus} (setminus)}{/}
\key{\disp{\cup} (cup)}{+}
\key{\disp{\cap} (cap)}{-}
\key{\disp{\langle} (langle)}{(}
\key{\disp{\rangle} (rangle)}{)}
\key{\disp{\exp} (exp)}{C-e}
\key{\disp{\sin} (sin)}{C-s}
\key{\disp{\cos} (cos)}{C-c}
\key{\disp{\sup} (sup)}{C-^}
\key{\disp{\inf} (inf)}{C-_}
\key{\disp{\det} (det)}{C-d}
\key{\disp{\lim} (lim)}{C-l}
\key{\disp{\tan} (tan)}{C-t}
\key{\disp{\hat{ }} (hat)}{^}
\key{\disp{\vee} (vee)}{v}

\section{Miscellaneous}

\key{cal letters}{c {\rm LETTER}}

\copyrightnotice

\bye

% End:

