% -*- plain-tex -*- 
% Reference Card for AUCTeX version 11.51
%**start of header
\newcount\columnsperpage

% This file has only been checked with 3 columns per page.  But it
% should print fine either via DVI or PDFTeX.

\columnsperpage=3

% Papersize stuff.  Use default paper size for PDF, but switch
% orientation.  Use papersize special for dvips.

\ifcase\ifx\pdfoutput\undefined 1 \fi
  \ifnum\pdfoutput=0 1 \fi 2
\or
%    \special{papersize 8.5in,11in}%
     \special{papersize 297mm,210mm}%
\or
      \dimen0\pdfpagewidth
      \pdfpagewidth\pdfpageheight
      \pdfpageheight\dimen0
\fi


% This file is intended to be processed by plain TeX (TeX82).
% compile-command: "tex tex-ref" or "pdftex tex-ref"
%
% Original author of Auc-TeX Reference Card:
%                                  
%       Terrence Brannon, PO Box 5027, Bethlehem, PA 18015 , USA
%  internet: tb06@pl118f.cc.lehigh.edu  (215) 758-1720 (215) 758-2104
%
% Kresten Krab Thorup updated the reference card to 6.
% Per Abrahamsen updated the reference card to 7, 8, and 9.
% Ralf Angeli updated it to 11.50.
% And David Kastrup messed around with it, too, merging the math reference.
%
% Thanks to Stephen Gildea
% Paul Rubin, Bob Chassell, Len Tower, and Richard Mlynarik
% for creating the GNU Emacs Reference Card from which this was mutated

\def\versionnumber{11.51}
\def\year{2004}
\def\version{August \year\ v\versionnumber}

\def\shortcopyrightnotice{\vskip 1ex plus 2 fill
  \centerline{\small \copyright\ \year\ Free Software Foundation, Inc.
  Permissions on back.  v\versionnumber}}

\def\copyrightnotice{%
\vskip 1ex plus 2 fill\begingroup\small
\centerline{Copyright \copyright\ 1987, 1993, 2004 Free Software Foundation, Inc.}
\centerline{Copyright \copyright\ 1992 Kresten Krab Thorup}
\centerline{for AUC\TeX\ version \versionnumber}

Permission is granted to make and distribute copies of
this card provided the copyright notice and this permission notice
are preserved on all copies.


\endgroup}

% make \bye not \outer so that the \def\bye in the \else clause below
% can be scanned without complaint.
\def\bye{\par\vfill\supereject\end}

\newdimen\intercolumnskip
\newbox\columna
\newbox\columnb

\edef\ncolumns{\the\columnsperpage}

\message{[\ncolumns\space 
  column\if 1\ncolumns\else s\fi\space per page]}

\def\scaledmag#1{ scaled \magstep #1}

% This multi-way format was designed by Stephen Gildea
% October 1986.
\if 1\ncolumns
  \hsize 4in
  \vsize 10in
  \voffset -.7in
  \font\titlefont=\fontname\tenbf \scaledmag3
  \font\headingfont=\fontname\tenbf \scaledmag2
  \font\smallfont=\fontname\sevenrm
  \font\smallsy=\fontname\sevensy

  \footline{\hss\folio}
  \def\makefootline{\baselineskip10pt\hsize6.5in\line{\the\footline}}
\else
  \hsize 3.2in
  \vsize 7.6in
  \hoffset -.75in
  \voffset -.8in
  \font\titlefont=cmbx10 \scaledmag2
  \font\headingfont=cmbx10 \scaledmag1
  \font\smallfont=cmr6
  \font\smallsy=cmsy6
  \font\eightrm=cmr8
  \font\eightbf=cmbx8
  \font\eightit=cmti8
  \font\eighttt=cmtt8
  \font\eightsl=cmsl8
  \font\eightsc=cmcsc8
  \font\eightsy=cmsy8
  \textfont0=\eightrm
  \textfont2=\eightsy
  \def\rm{\fam0 \eightrm}
  \def\bf{\eightbf}
  \def\it{\eightit}
  \def\tt{\eighttt}
  \def\sl{\eightsl}
  \def\sc{\eightsc}
  \normalbaselineskip=.8\normalbaselineskip
  \ht\strutbox.8\ht\strutbox
  \dp\strutbox.8\dp\strutbox
  \normallineskip=.8\normallineskip
  \normallineskiplimit=.8\normallineskiplimit
  \normalbaselines\rm           %make definitions take effect

  \if 2\ncolumns
    \let\maxcolumn=b
    \footline{\hss\rm\folio\hss}
    \def\makefootline{\vskip 2in \hsize=6.86in\line{\the\footline}}
  \else \if 3\ncolumns
    \let\maxcolumn=c
    \nopagenumbers
  \else
    \errhelp{You must set \columnsperpage equal to 1, 2, or 3.}
    \errmessage{Illegal number of columns per page}
  \fi\fi

  \intercolumnskip=.46in
  \def\abc{a}
  \output={%
      % This next line is useful when designing the layout.
      %\immediate\write16{Column \folio\abc\space starts with \firstmark}
      \if \maxcolumn\abc \multicolumnformat \global\def\abc{a}
      \else\if a\abc
        \global\setbox\columna\columnbox \global\def\abc{b}
        %% in case we never use \columnb (two-column mode)
        \global\setbox\columnb\hbox to -\intercolumnskip{}
      \else
        \global\setbox\columnb\columnbox \global\def\abc{c}\fi\fi}
  \def\multicolumnformat{\shipout\vbox{\makeheadline
      \hbox{\box\columna\hskip\intercolumnskip
        \box\columnb\hskip\intercolumnskip\columnbox}
      \makefootline}\advancepageno}
  \def\columnbox{\leftline{\pagebody}}

  \def\bye{\par\vfill\supereject
    \if a\abc \else\null\vfill\eject\fi
    \if a\abc \else\null\vfill\eject\fi
    \end}  
\fi

% we won't be using math mode much, so redefine some of the characters
% we might want to talk about
\catcode`\^=12
\catcode`\_=12

\chardef\\=`\\
\chardef\{=`\{
\chardef\}=`\}

\hyphenation{mini-buf-fer}

\parindent 0pt
\parskip 1ex plus .5ex minus .5ex

\def\small{\smallfont\textfont2=\smallsy\baselineskip=.8\baselineskip}

\outer\def\newcolumn{\vfill\eject}

\outer\def\title#1{{\titlefont\centerline{#1}}\vskip 1ex plus .5ex}

\outer\def\section#1{\par\vskip 0pt plus 0.2\vsize \penalty-3000
         \vskip 0pt plus -0.2\vsize
  \vskip 3ex plus 2ex minus 2ex {\headingfont #1}\mark{#1}%
  \vskip 2ex plus 1ex minus 1.5ex}

\newdimen\keyindent

\def\beginindentedkeys{\keyindent=1em}
\def\endindentedkeys{\keyindent=0em}
\endindentedkeys

\def\paralign{\vskip\parskip\halign}

\def\<#1>{$\langle${\rm #1}$\rangle$}

\def\kbd#1{{\tt#1}\null}        %\null so not an abbrev even if period follows

\def\beginexample{\par\leavevmode\begingroup
  \obeylines\obeyspaces\parskip0pt\tt}
{\obeyspaces\global\let =\ }
\def\endexample{\endgroup}

\def\key#1#2{\leavevmode\hbox to \hsize{\vtop
  {\hsize=.68\hsize\rightskip=1em
  \hskip\keyindent\relax#1}\kbd{#2}\hfil}}

\newbox\metaxbox
\setbox\metaxbox\hbox{\kbd{M-x }}
\newdimen\metaxwidth
\metaxwidth=\wd\metaxbox

\def\metax#1#2{\leavevmode\hbox to \hsize{\hbox to .75\hsize
  {\hskip\keyindent\relax#1\hfil}%
  \hskip -\metaxwidth minus 1fil
  \kbd{#2}\hfil}}

\def\threecol#1#2#3{\hskip\keyindent\relax#1\hfil&\kbd{#2}\quad
  &\kbd{#3}\quad\cr}

\def\LaTeX{%
    L\kern-.36em\raise.3ex\hbox{\sc{a}}\kern-.15em\TeX}

%**end of header

\title{AUC\TeX\ Reference Card}

\centerline{(for version \versionnumber)}

\section{Conventions Used}

\key{Carriage Return or \kbd{C-m}}{RET}
\key{Tabular or \kbd{C-i}}{TAB}
\key{Linefeed or \kbd{C-j}}{LFD}

\section{Shell Interaction}

\key{Run a command on the master file}{C-c C-c}
\key{Run a command on the buffer}{C-c C-b}
\key{Run a command on the region}{C-c C-r}
\key{Fix the region}{C-c C-t C-r}
\key{Kill job}{C-c C-k}
\key{Recenter output buffer}{C-c C-l}
\key{Next error in \TeX/\LaTeX\ session}{C-c `}
\key{Toggle debug of wonderful boxes}{C-c C-w} % wonderful?
\key{View output file}{C-c C-v}

Commands you can run on the master file (with \kbd{C-c C-c}) or the
region (with \kbd{C-c C-r}) include the following (starred versions
are not available in all modes):

\def\star{\llap{\rm*}}
\key{\TeX}{\star TeX}
\key{\LaTeX}{\star LaTeX}
\key{Con\TeX{}t (once)}{\star ConTeXt}
\key{Con\TeX{}t Full}{\star ConTeXt Full}
\key{Makeinfo}{\star Makeinfo}
\key{Makeinfo with HTML output}{\star Makeinfo HTML}
\key{Appropriate previewer}{View}
\key{Print the output}{Print}
\key{Bib\TeX}{BibTeX}
\key{MakeIndex}{Index}
\key{LaCheck}{Check}
\key{Make (PostScript) File}{File}
\key{Ispell}{Spell}

\section{\TeX ing options}
\TeX\ runs can come in various types, which may be toggled and are
indicated in the mode line.

\key{$\Omega$ mode}{C-c C-t C-o}
\key{PDF/DVI mode}{C-c C-t C-p}
\key{Stop on errors (Interactive mode)}{C-c C-t C-i}
\key{Use Source Specials for viewer control}{C-c C-t C-s}

\section{Miscellaneous}

\key{Read AUC\TeX\ manual}{C-c TAB}
\key{Math Mode}{C-c \string~}
\key{Reset Buffer}{C-c C-n}
\key{Reset AUC\TeX}{C-u C-c C-n}

\section{Multifile Handling}

\key{Save Document}{C-c C-d}
\key{Switch to master file or active buffer}{C-c ^}
\key{Query for a master file}{C-c \_}

\section{Command Insertion}

\key{Insert Section}{C-c C-s}
\key{Insert \LaTeX\ environment}{C-c C-e}
\key{Insert item}{C-c LFD}
\key{Insert item (alias)}{M-RET}
\key{Close \LaTeX\ environment}{C-c ]}
\key{Insert \TeX\ macro \kbd{\\\{\}} }{C-c C-m}
\key{Insert double brace}{C-c \{}
\key{Complete \TeX\ macro}{M-TAB}
\key{Smart ``quote''}{"}
\key{Smart ``dollar''}{\$}
  
\section{Font Selection}

\key{Insert {\bf bold\/} text}{C-c C-f C-b}
\key{Insert {\it italics\/} text}{C-c C-f C-i}
\key{Insert {\rm roman} text}{C-c C-f C-r}
\key{Insert {\it emphasized\/} text}{C-c C-f C-e}
\key{Insert {\tt typewriter\/} text}{C-c C-f C-t}
\key{Insert {\sl slanted\/} text}{C-c C-f C-s}
\key{Insert {\sc Small Caps\/} text}{C-c C-f C-c}
\key{Delete font}{C-c C-f C-d}
\key{Replace font}{C-u C-c C-f \<key>}

\section{Source Formatting}

\key{Indent current line}{TAB}
\key{Indent next line}{LFD}

\key{Format a paragraph}{M-q}
\key{Format a region}{C-c C-q C-r}
\key{Format a section}{C-c C-q C-s}
\key{Format an environment}{C-c C-q C-e}

\key{Mark an environment}{C-c .}
\key{Mark a section}{C-c *}

\key{Comment or uncomment region}{C-c ;}
\key{Comment or uncomment paragraph}{C-c \%}

\section{Source Display}

\key{Toggle folding mode}{C-c C-o C-f}
\key{Hide all items in buffer}{C-c C-o C-b}
\key{Hide current macro}{C-c C-o C-m}
\key{Hide current environment}{C-c C-o C-e}
\key{Show all items in buffer permanently}{C-c C-o C-x}
\key{Show current item permanently}{C-c C-o C-c}

\copyrightnotice
\penalty-9000
\title{Math Mode}

\section{Variables}

All math mode commands are under the prefix key specified by
LaTeX-math-abbrev-prefix, default is "`".

You can define your own math mode commands by setting the variable
LaTeX-math-list before loading LaTeX-math-mode.

\section{Greek Letters}

\def\disp#1{\hbox to 6ex{$#1$\hfill}}
\def\twocol#1\par{{%
  \def\key##1##2{##1&##2\cr}%
  \setbox0\vbox{\halign to 0.45\hsize{\tabskip0ptplus1fil\relax
    ##\hfil&\kbd{##}\hfil\cr\vrule width0ptheight\ht\strutbox#1}}%
  \line{%
  \splittopskip=\ht\strutbox
  \dimen0\ht0
  \advance\dimen0\baselineskip
  \setbox2\vsplit0to0.5\dimen0
  \vtop{\unvbox2}\hfill\raise \ht\strutbox \vtop {\unvbox0}}}}
\def\keycs#1#2#{\keycsii#1{#2}}
\def\keycsii#1#2#3{\key{\disp{#1#2} ({\tt\string#1})}{#3}}

\twocol
\keycs\alpha{a}
\keycs\beta{b}
\keycs\gamma{g}
\keycs\delta{d}
\keycs\epsilon{e}
\keycs\zeta{z}
\keycs\eta{h}
\keycs\theta{j}
\keycs\kappa{k}
\keycs\lambda{l}
\keycs\mu{m}
\keycs\nu{n}
\keycs\xi{x}
\keycs\pi{p}
\keycs\rho{r}
\keycs\sigma{s}
\keycs\tau{t}
\keycs\upsilon{u}
\keycs\phi{f}
\keycs\chi{q}
\keycs\psi{y}
\keycs\omega{w}
\keycs\Delta{D}
\keycs\Gamma{G}
\keycs\Theta{Q}
\keycs\Lambda{L}
\keycs\Pi{P}
\keycs\Sigma{S}
\keycs\Upsilon{U}
\keycs\Phi{F}
\keycs\Psi{Y}
\keycs\Omega{W}

\section{Symbols}

\twocol
\keycs\rightarrow{C-f}
\keycs\leftarrow{C-b}
\keycs\uparrow{C-p}
\keycs\downarrow{C-n}
\keycs\leq{<}
\keycs\geq{>}
\keycs\tilde x{\string~}
\keycs\nabla{N}
\keycs\infty{I}
\keycs\forall{A}
\keycs\exists{E}
\keycs\not \ {/}
\keycs\in{i}
\keycs\times{*}
\keycs\cdot{.}
\keycs\subset{\{}
\keycs\supset{\}}
\keycs\subseteq{[}
\keycs\supseteq{]}
\keycs\emptyset{0}
\keycs\setminus{\\}
\keycs\cup{+}
\keycs\cap{-}
\keycs\langle{(}
\keycs\rangle{)}
\keycs\exp{C-e}
\keycs\sin{C-s}
\keycs\cos{C-c}
\keycs\sup{C-^}
\keycs\inf{C-_}
\keycs\det{C-d}
\keycs\lim{C-l}
\keycs\tan{C-t}
\keycs\hat x{^}
\keycs\vee{|}
\keycs\wedge{\&}

\section{Miscellaneous}

\key{cal letters}{c \<letter>}

\bye

%%% Local Variables: 
%%% mode: plain-TeX
%%% TeX-master: t
%%% End: 
