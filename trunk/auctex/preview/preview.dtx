%\CheckSum{398}
%\def\next#1Name: #2${\def\next{#2}}
%\next $Name: not supported by cvs2svn $
%\ifx\next\empty
%  \def\next#1Revision: #2${\def\version{CVS-#2}}
%  \next $Revision: 1.18 $
%\else
%  \def\version rel-#2 {\def\version{#2}}
%  \expandafter\version\next 
%  \begingroup \lccode`-=`.
%  \edef\next{\lowercase{\endgroup\def\noexpand\version{\next}}}
%  \next
%\fi
%\def\next#1Date: #2${\date{#2}}
%\next $Date: 2001-10-16 21:31:44 $
%\author{David Kastrup\thanks{\texttt{dakas@users.sourceforge.net}}}
%\title{The \texttt{preview} Package for \LaTeX\\Version \version}
%\maketitle
%\section{Introduction}
%The main purpose of this package is the extraction of certain
%environments (most notably displayed formulas) for use in different
%contexts.  While the erstwhile application has been the embedding of
%those preview fragments into Emacs source buffers under the Auc\TeX\
%editing environment, other applications are easily imaginable.
%
%In particular it should be noted that producing EPS files with
%Dvips and its derivatives using the \texttt{-E} option is
%not currently well-supported by \LaTeX.  People make do by fiddling
%around with |\thispagestyle{empty}| and hoping for the best (namely,
%that the specified contents will indeed fit on single pages), and
%then trying to guess the baseline of the resulting code and stuff,
%but this is at best dissatisfactory.  The preview package provides an
%easy way to ensure that exactly one page per request gets shipped,
%with a well-defined baseline and no page decorations.  Thus you can
%safely use
%\begin{quote}
%|dvips -E -i|
%\end{quote}
%and get a single EPS file with shrink-wrapped bounding box for every
%generated image of a LaTeX run.  The possibility for generating a
%whole set of graphics with a single run of \LaTeX\ and Dvips
%increases both speed and robustness of applications.  It is to be
%hoped that applications like \LaTeX HTML will be able to make use of
%this package in future.  For this reason, once it stabilizes, it will
%be submitted to CTAN.
%
%\section{Package options}
%The package is included with the customary
%\begin{quote}
%\cmd\usepackage\oarg{options}|{preview}|
%\end{quote}
%You should usually load this package as the last one, since it
%redefines several things that other packages may also provide.
%
%The following options are available:
%\begin{description}
%\item[|active|] is the most essential option.  If this option is not
%specified, the |preview| package will be inactive and the document
%will be typeset as if the |preview| package were not loaded, except
%that all declarations and environments defined by the package are
%still legal but have no effect.  This allows defining previewing
%characteristics in your document, and only activating them by calling
%\LaTeX\ as
%\begin{quote}
%|latex '\PassOptionsToPackage{active}{preview}| |\input|\marg{filename}|'|
%\end{quote}
%\item[|displaymath|] will make all displayed math environments subject to
%preview processing.  This will typically be the most desired option.
%\item[|floats|] will make all float objects subject to preview
%processing.  If you want to be more selective about what floats to
%pass through to a preview, you should instead use the
%\cmd{\PreviewSnarfEnvironment} command on the floats you want to have
%previewed.
%\item[|textmath|] will make all text math subject to previews.  Since
%math mode is used throughly inside of LaTeX even for other purposes,
%this works by redefining \cmd\(, \cmd\) and |$|.  Only occurences of
%these text math delimiters in later loaded packages and in the main
%document will thus be affected.
%\item[|graphics|] will subject all \cmd{\includegraphics} commands to
%a preview.
%\item[|sections|] will subject all section headers to a preview.
%\item[|delayed|] will delay all activations and redefinitions the
%|preview| package makes until |\begin{document}|.  The purpose of
%this is to cater for documents which should be subjected to the
%|preview| package without having been prepared for it.  You can
%process such documents with
%\begin{quote}
%|latex '\RequirePackage[active,delayed,|\meta{options}|]{preview}|
%|\input|\marg{filename}|'|
%\end{quote}
%This relaxes the requirement to be loading the |preview| package as
%last package.
%\item[\meta{driver}] loads a special driver file
%|pr|\meta{driver}|.def|.  Currently the only supported driver is
%|auctex|.  This driver will produce fake error messages at the start and
%end of every preview environment that enable a special Emacs package in
%connection with Auc\TeX\ to pinpoint the exact source location where
%the previews have originated.  Unfortunately, there is no other
%reliable means of passing the current \TeX\ input position
%\emph{in} a line to external programs.
%You should not specify this option manually, since it will only be
%needed by automated runs that want to parse the pseudo error
%messages.  Those runs will then use \cmd{\PassOptionsToPackage} in
%order to effect the desired behaviour.
%\end{description}
%\section{Provided Commands}
%\DescribeEnv{preview} The |preview| environment causes its contents to
%be set as a single preview image.  Insertions like figures and
%footnotes (except those
%included in minipages) will typically lead to error messages or
%be lost.  In case the |preview| package has not been activated, the
%contents of this environment will be typeset normally.
%
%\DescribeEnv{nopreview} The |nopreview| environment will cause its
%contents not to undergo any special treatment by the |preview|
%package.  When |preview| is active, the contents will be discarded
%like all main text that does not trigger the |preview| hooks.  When
%|preview| is not active, the contents will be typeset just like the
%main text.
%
%Note that both of these environments typeset things when preview is
%not active.  If you need something typeset conditionally, use the
%\cmd{\ifPreview} conditional for it.
%
%\DescribeMacro{\PreviewMacro} If you want to make a macro like
%\cmd{\includegraphics} (actually, this is what is done by the
%|graphics| option to |preview|) produce a preview image, you put a
%declaration like
%\begin{quote}
%|\PreviewMacro[*[[!]{\includegraphics}|
%\end{quote}
%or, more readable,
%\begin{quote}
%|\PreviewMacro[{*[][]{}}]{\includegraphics}|
%\end{quote}
%into your preamble.  The optional argument to \cmd{\PreviewMacro}
%specifies the arguments \cmd{\includegraphics} accepts, since this is
%necessary information for properly ending the preview box.  Note that
%if you are using the more readable form, you have to enclose the
%argument in a |[{| and |}]| pair.  The inner braces are necessary to
%stop any included |[]| pairs from prematurely ending the optional
%argument, and to make a single |{}| denoting an optional argument not
%get stripped away by \TeX's argument parsing.
%The
%letters simply mean
%\begin{description}
%\item[|*|] indicates an optional |*| modifier, as in
%|\includegraphics*|.
%\item[|[|] indicates an optional argument in brackets.  This syntax
%is somewhat disingenious, but brief.
%\item[{|[]|}] also indicates an optional argument in brackets.  Be
%sure to have encluded the entire optional argument specification in
%an additional pair of braces as described above.
%\item[|!|] indicates a mandatory argument.
%\item[|\char`{\char`}|] indicates the same.  Again, be sure to have that
%additional level of braces around the whole argument specification.
%\end{description}
%
%\DescribeMacro{\PreviewEnv} The macro \cmd{\PreviewEnv} works just as
%\cmd{\PreviewMacro} does, only for environments.
%
%\DescribeMacro{\PreviewSnarfEnv} This macro does not typeset the
%original environment inside of a preview box, but instead typesets
%just the contents of the original environment inside of the preview
%box, leaving nothing for the original environment.  This has to be
%used for figures, for example, since they would
%\begin{enumerate}
%\item produce insertion material that cannot be extracted to the
%preview properly,
%\item complain with an error message about not being in outer par
%mode.
%\end{enumerate}
%
%\DescribeMacro{\ifPreview} In case you need to know whether |preview|
%is active, you can use the conditional \cmd{\ifPreview} together with
%|\else| and |\fi|.  
%\StopEventually{}
%\section{The Implementation}
% Here we go:
%    \begin{macrocode}
%<*style>
%<*!active>
\NeedsTeXFormat{LaTeX2e}
\ProvidesPackage{preview}[2001/10/05 version 0.5+]
%    \end{macrocode}
%Since many parts here will not be needed as long as the package is
%inactive, we will include them enclosed with |<*active>| and
%|</active>| guards.  That way, we can append all of this stuff at a
%place where it does not get loaded if not necessary.
%
%\begin{macro}{\ifPreview}
%  Setting the \cmd{\ifPreview} command should not be done by the user,
%  so we don't use \cmd{\newif} here.  As a consequence, there are no
%  \cmd{\Previewtrue} and \cmd{\Previewfalse} commands.
%    \begin{macrocode}
\let\ifPreview\iffalse
%</!active>
%    \end{macrocode}
%\end{macro}
%\begin{macro}{\ifpr@outer}
%  We don't allow previews inside of previews.  The macro
%  \cmd{\ifpr@outer} can be used for checking whether we are outside of
%  any preview code.
%    \begin{macrocode}
%<*active>
\newif\ifpr@outer
\pr@outertrue
%</active>
%    \end{macrocode}
%\end{macro}
%
%\begin{macro}{\preview@delay}
%  The usual meaning of \cmd{\preview@delay} is to just echo its
%  argument in normal |preview| operation.  If |preview| is inactive, it
%  swallows its argument.  If the |delayed| option is active, the
%  contents will be passed to the \cmd{\AtBeginDocument} hook.
%\begin{macro}{\pr@advise}
%  The core macro for modifying commands is \cmd{\pr@advise}.  You
%  pass it the original command name as first argument and the new
%  definition as a second argument, with |#1| in the place where the
%  original argument should be called.  The new definition gets
%  \cmd{\edef}-expanded, so prepare it properly.
%\begin{macro}{\pr@advise@ship}
%  The most often used macro for modifying commands is
%  \cmd{\pr@advise@ship}.  It receives three arguments.  The first is
%  the macro to modify, the second specifies some actions to be done
%  inside of a box to be created before the original macro gets
%  executed, the third one specifies actions after the original macro
%  got executed.
%    \begin{macrocode}
%<*!active>
\let\preview@delay=\@gobble
\let\pr@advise=\@gobbletwo
\def\pr@advise@ship#1#2#3{}
%    \end{macrocode}
%  These commands get more interesting when |preview| is active:
%    \begin{macrocode}
\DeclareOption{active}{%
  \let\ifPreview\iftrue
  \def\pr@advise#1{%
    \expandafter\pr@adviseii\csname pr@\string#1\endcsname#1}%
  \def\pr@advise@ship#1#2#3{\toks@{\pr@protect@ship{#2}{#3}}%
     \pr@advise#1{\the\toks@\noexpand##1}}%
  \let\preview@delay\@firstofone}
%    \end{macrocode}
%\end{macro}
%\end{macro}
%\end{macro}
%\begin{macro}{\pr@adviseii}
%  Now \cmd{\pr@advise} needs its helper macro.  In order to avoid
%  recursive definitions, we advise only macros that are not yet
%  advised.  Or, more exactly, we throw away the old advice and only
%  take the new one.
%    \begin{macrocode}
\def\pr@adviseii#1#2#3{\preview@delay{%
  \ifx#1\relax \let#1#2\fi
  \def\reserved@a##1{#3}%
  \edef#2{\reserved@a#1}}}
%    \end{macrocode}
%\end{macro}
%
%  The |delayed| option is easy to implement:
%    \begin{macrocode}
\DeclareOption{delayed}{%
  \ifPreview \let\preview@delay\AtBeginDocument \fi
}
%</!active>
%    \end{macrocode}
%
%\subsection{The internals}
%
%Those are only needed if |preview| is active.
%    \begin{macrocode}
%<*active>
%    \end{macrocode}
%\begin{macro}{\pr@snippet}
%  \cmd{\pr@snippet} is the current snippet number.  We need a
%  separate counter to \cmd{\c@page} since several other commands
%  might fiddle with the page number.
%    \begin{macrocode}
\newcount\pr@snippet
\global\pr@snippet=1
%    \end{macrocode}
%\end{macro}
%\begin{macro}{\pr@protect@ship}
%  Let's begin with the above mentioned \cmd{\pr@protect@ship}.  This
%  gets three arguments.  The first is what to do at the beginning of
%  the preview, the second what to do at the end, the third is the
%  macro where we stored the original definition.
%
%  In case we are not in a typesetting situation,
%  \cmd{\pr@protect@ship} leaves the stored macro to fend for its
%  own.  No better or worse protection than the original.  And we only
%  do anything different when \cmd{\ifpr@outer} turns out to be true.
%    \begin{macrocode}
\def\pr@protect@ship{\ifx\protect\@typeset@protect
   \expandafter\pr@startbox\else \expandafter\@gobbletwo \fi}
%    \end{macrocode}
%\end{macro}
%\begin{macro}{\pr@box}
%\begin{macro}{\pr@startbox}
%  Previews will be stored in \cmd{\box}\cmd{\pr@box}.
%  \cmd{\pr@startbox} gets two arguments: code to execute immediately
%  before the following stuff, code to execute afterwards.  You have
%  to cater for \cmd{\pr@endbox} being called at the right time
%  yourself.
%    \begin{macrocode}
\newbox\pr@box
\def\pr@startbox#1#2{%
  \ifpr@outer
    \pr@ship@start
    \toks@{#2}%
    \edef\pr@cleanup{\the\toks@}%
    \setbox\pr@box\vbox\bgroup
    \pr@outerfalse\color@setgroup
    \everypar{}%
    \expandafter\@firstofone
  \else
     \expandafter \@gobble
  \fi{#1}}
%    \end{macrocode}
% \cmd{\everypar} is cleared here because of a surprising bug where an
% \cmd{\item} label crept into a preview following it.
%\end{macro}
%\end{macro}
%\begin{macro}{\pr@endbox}
%  Cleaning up also is straightforward.
%    \begin{macrocode}
\def\pr@endbox{%
   \color@endgroup
   \egroup
%    \end{macrocode}
%  Except that we shift the output box up by its height.  This has two
%  purposes: first we make sure that no package-inflicted non-zero
%  value of \cmd{\voffset} or \cmd{\hoffset} will have any influence
%  of the positioning of our box.  Second we shift the box such that
%  its basepoint will exactly be at the (1in,1in)~mark defined by
%  TeX.  That way we can properly take ascenders into account.
%    \begin{macrocode}
   {\let\protect\noexpand
   \voffset=-\ht\pr@box
   \hoffset=\z@
   \c@page=\pr@snippet
   \shipout\box\pr@box}%
   \pr@ship@end
   \global\advance\pr@snippet\@ne
   \pr@cleanup
}
%    \end{macrocode}
%\end{macro}
%\subsection{Parsing commands}
%\begin{macro}{\pr@callafter}
%\begin{macro}{\pr@parseit}
%\begin{macro}{\pr@seq}
%\begin{macro}{\pr@endparse}
%  The following stuff is for parsing the arguments of commands we want
%  to somehow surround with stuff.  Usage is
%    \begin{quote}
%    \cmd{\pr@callafter}\meta{aftertoken}\meta{parsestring}\cmd{\pr@endparse}\\
%    \qquad\meta{macro}\meta{parameters}
%    \end{quote}
%  \meta{aftertoken} is stored in \cmd{\pr@seq} and gets placed after
%  the completely parsed macro.  \meta{parsestring} would be, for
%  example for the \cmd{\includegraphics} macro, |*[[!|, an optional
%  |*| argument followed by two optional arguments enclosed in |[]|,
%  followed by one mandatory argument.
%
%  For the sake of a somewhat more intuitive syntax, we now support
%  also the syntax |{*[]{}}| in the optional argument.  Since \TeX\
%  strips redundant braces, we have to write |[{{}}]| in this syntax
%  for a single mandatory argument.  Hard to avoid.  We use an unusual
%  character for ending the parsing.
%  The implementation is rather trivial.
%    \begin{macrocode}
\def\pr@callafter{%
  \afterassignment\pr@parseit
  \let\pr@seq= }
\def\pr@parseit#1{\csname pr@parse#1\endcsname}
\let\pr@endparse=\@percentchar
%    \end{macrocode}
%\end{macro}
%\end{macro}
%\end{macro}
%\end{macro}
%\begin{macro}{\pr@parse*}
%  Straightforward, same mechanism \LaTeX\ itself employs.
%    \begin{macrocode}
\expandafter\def\csname pr@parse*\endcsname#1\pr@endparse#2{%
  \@ifstar{\pr@parseit#1\pr@endparse{#2*}}%
          {\pr@parseit#1\pr@endparse{#2}}}
%    \end{macrocode}
%\end{macro}
%\begin{macro}{\pr@parse[}
%\begin{macro}{\pr@brace}
%  Copies optional parameters in brackets if present.  The additional
%  level of braces is necessary to ensure that braces the user might
%  have put to hide a~|]| bracket in an optional argument don't get
%  lost.  There will be no harm if such braces were not there at the
%  start.
%    \begin{macrocode}
\expandafter\def\csname pr@parse[\endcsname#1\pr@endparse#2{%
  \@ifnextchar[{\pr@bracket#1\pr@endparse{#2}}%
               {\pr@parseit#1\pr@endparse{#2}}}
\def\pr@bracket#1\pr@endparse#2[#3]{\pr@parseit#1\pr@endparse{#2[{#3}]}}
%    \end{macrocode}
%\end{macro}
%\end{macro}
%\begin{macro}{\pr@parse]}
%  This is basically a do-nothing, so that we may use the syntax
%  |{*[][]!}| in the optional argument instead of the more concise but
%  ugly |*[[!| which confuses the brace matchers of editors.
%    \begin{macrocode}
\expandafter\let\csname pr@parse]\endcsname=\pr@parseit
%    \end{macrocode}
%\end{macro}
%\begin{macro}{\pr@parse}
%\begin{macro}{\pr@parse!}
%  Mandatory arguments are perhaps easiest to parse.
%    \begin{macrocode}
\def\pr@parse#1\pr@endparse#2#3{%
  \pr@parseit#1\pr@endparse{#2{#3}}}
\expandafter\let\csname pr@parse!\endcsname=\pr@parse
%    \end{macrocode}
%\end{macro}
%\end{macro}
%\begin{macro}{\pr@parse\pr@endparse}
%  And finally the macro that gets called at the end and wraps all
%  this up by placing the completed macro call and then putting
%  \cmd{\pr@seq} behind it.
%    \begin{macrocode}
\expandafter\def\csname pr@parse\pr@endparse\endcsname#1{#1\pr@seq}
%</active>
%    \end{macrocode}
%\end{macro}
%\subsection{Selection options}
%  The |displaymath| option.
%    \begin{macrocode}
%<*!active>
\DeclareOption{displaymath}{%
  \preview@delay{%
    \everydisplay\expandafter{\the\everydisplay
    \pr@startbox{$$\aftergroup\pr@endbox}{$$}}}%
}
%    \end{macrocode}
%
%  The |textmath| option.  Some folderol in order to define the active
%  |$| math mode delimiter.
%    \begin{macrocode}
\begingroup
\def\next#1#2{%
\endgroup
    \DeclareOption{textmath}{%
      \preview@delay{\ifx#1\@undefined \let#1=$\fi
        \catcode`\$=\active}%
      \pr@advise@ship\(\pr@endaftergroup{}% \)
      \pr@advise@ship#1{\let#1=#2%
         \pr@setmathhooks{##1\everymath##1\everydisplay}}{}}}
\lccode`\~=`\$
\lowercase{\expandafter\next\expandafter~}%
  \csname pr@\string$\endcsname
%</!active>
%    \end{macrocode}
%\begin{macro}{\pr@endaftergroup}
%  This justs ends the box after the group opened by |#1| is closed
%  again.
%    \begin{macrocode}
%<*active>
\def\pr@endaftergroup#1{#1\aftergroup\pr@endbox}
%    \end{macrocode}
%\end{macro}
%\begin{macro}{\pr@setmathhooks}
%  This is here for making sure display math entered by |$$| will work
%  more or less as expected (except of getting activated by the
%  |textmath| option).  While |$$| is not to be used in \LaTeX\
%  because it will cause inconsistent spacing compared to the other
%  environments, we don't want to get bugged with error reports caused
%  by |$$| just being an empty text math environment.
%  Since we have restored the original meaning of the token we are
%  going to parse, we have here a regular math environment enterer.
%  We load both \cmd{\everymath} and \cmd{\everydisplay} hooks with
%  code that restores both hooks to the previous state, then does
%  \begin{quote}
%    |\aftergroup \pr@endbox|
%  \end{quote}
%  and then executes the restored hook.  This code is to be relished,
%  not explained.
%    \begin{macrocode}
\def\pr@setmathhooks#1{\def\reserved@a##1{#1}%
  \def\reserved@b##1{##1{\the##1}}%
  \def\reserved@c##1{%
    ##1{\reserved@a\reserved@b\aftergroup\noexpand\pr@endbox
        \noexpand\the##1}}
   \edef\reserved@a{\reserved@a\reserved@c}\reserved@a}
%</active>
%    \end{macrocode}
%\end{macro}
%  The |graphics| option.
%    \begin{macrocode}
%<*!active>
\DeclareOption{graphics}{%
  \PreviewMacro[{*[[!}]{\includegraphics}}
%    \end{macrocode}
%  The |floats| option.
%    \begin{macrocode}
\DeclareOption{floats}{%
  \PreviewSnarfEnvironment[![]{@float}%
  \PreviewSnarfEnvironment[![]{@dblfloat}}
%    \end{macrocode}
%  The |sections| option.
%    \begin{macrocode}
\DeclareOption{sections}{%
  \PreviewMacro[!!!!!!*[!]{\@startsection}}
%    \end{macrocode}
%We now interpret any further options as driver files we load.
%    \begin{macrocode}
\DeclareOption*
   {\InputIfFileExists{pr\CurrentOption.def}{}{\OptionNotUsed}}
%    \end{macrocode}
%
%\subsection{Preview attaching commands}
%  \begin{macro}{\PreviewMacro}
%  As explained above.
%    \begin{macrocode}
\newcommand\PreviewMacro[2][]{%
   \pr@advise@ship{#2}{\pr@callafter\pr@endbox#1\pr@endparse}{}}
%    \end{macrocode}
%\end{macro}
%\begin{macro}{\PreviewEnvironment}
%  Actually, this ignores any syntax argument.  But don't tell
%  anybody.
%    \begin{macrocode}
\newcommand{\PreviewEnvironment}[2][]{\expandafter\pr@advise@ship
   \csname #2\endcsname{\aftergroup\pr@endbox}{}}
%    \end{macrocode}
%\end{macro}
%\begin{macro}{\PreviewSnarfEnvironment}[2][]
%  This is a nuisance since we have to advise \emph{both} the
%  environment and its end.
%    \begin{macrocode}
\newcommand{\PreviewSnarfEnvironment}[2][]{\toks@{%
    \pr@snarfafter#1\pr@endparse}%
  \expandafter\pr@advise
   \csname #2\endcsname{\the\toks@\noexpand##1}%
 \expandafter\pr@advise
   \csname end#2\endcsname{\endgroup\noexpand##1}}
%</!active>
%    \end{macrocode}
%\end{macro}
%\begin{macro}{\pr@snarfafter}
%\begin{macro}{\pr@startsnarf}
%  Ok, this looks complicated, but we have to start a group in order
%  to be able to hook \cmd{\pr@endbox} into the game only when
%  \cmd{\ifpr@outer} has triggered the start.  And we need to get our
%  start messages out before parsing the arguments.
%    \begin{macrocode}
%<*active>
\def\pr@snarfafter{\ifpr@outer
     \pr@ship@start
     \let\pr@ship@start\@empty
   \fi
  \pr@callafter\pr@startsnarf}
\def\pr@startsnarf{\begingroup
   \pr@startbox{\begingroup\aftergroup\pr@endbox}{\endgroup}%
   \ignorespaces}
%</active>
%    \end{macrocode}
%\end{macro}
%\end{macro}
%\begin{macro}{\pr@ship@start}
%\begin{macro}{\pr@ship@end}
%  The hooks \cmd{\pr@ship@start} and \cmd{\pr@ship@end} can be added
%  to by option files by the help of the \cmd{\g@addto@macro} command
%  from \LaTeX.  They are called just before starting to process some
%  preview, and just after it.
%    \begin{macrocode}
%<*!active>
\let\pr@ship@start\@empty
\let\pr@ship@end\@empty
%</!active>
%</style>
%    \end{macrocode}
%\end{macro}
%\end{macro}
%The Auc\TeX\ option will cause error messages to spew.  We want them
%on the terminal, but we don't want LaTeX to stop its automated run.
%We delay \cmd{\nonstopmode} in case the user has any
%pseudo-interactive folderol like reading in of file names in his
%preamble.  Because we are so good-hearted, we will not break this as
%long as the document has not started, but after that we need the
%error message mechanism operative.
%
%So here is the contents of the |prauctex.def| file:
%    \begin{macrocode}
%<auctex>\ifPreview
%<auctex>\preview@delay{\nonstopmode}
%<auctex>\def\pr@msg#1{\PackageError{Preview}%
%<auctex>   {Snippet \number\pr@snippet\space#1}%
%<auctex>   {There is nothing to be seen here.  What are you doing here?^^J%
%<auctex>    Just hit Return and be gone.}}
%<auctex>\g@addto@macro\pr@ship@start{\pr@msg{started}}
%<auctex>\g@addto@macro\pr@ship@end{\pr@msg{ended}}
%    \end{macrocode}
%Since over/underfull boxes are about the messiest things to parse, we
%disable them by setting the appropriate badness limits and making the
%variables point to junk.
%    \begin{macrocode}
%<auctex>\hbadness=10000
%<auctex>\vbadness=10000
%<auctex>\newcount\hbadness
%<auctex>\newcount\vbadness
%<auctex>\fi
%    \end{macrocode}
%
%\begin{environment}{preview}
%\begin{environment}{nopreview}
%  First we write the definitions of these environments when |preview|
%  is inactive.  We will redefine them if |preview| gets activated.
%    \begin{macrocode}
%<*style>
%<*!active>
\newenvironment{preview}{\ignorespaces}{\unskip}
\newenvironment{nopreview}{\ignorespaces}{\unskip}
%    \end{macrocode}
%\end{environment}
%\end{environment}
%
%We now process the options and finish in case we are not active.
%    \begin{macrocode}
\ProcessOptions\relax
\ifPreview\else\expandafter\endinput\fi
%</!active>
%    \end{macrocode}
%Now for the redefinition of the |preview| and |endpreview|
%environments:
%    \begin{macrocode}
%<*active>
\renewenvironment{preview}{\begingroup
   \pr@startbox{\begingroup\aftergroup\pr@endbox}{\endgroup}\ignorespaces}%
   {\unskip\endgroup}
\renewenvironment{nopreview}{\pr@outerfalse\ignorespaces}{\unskip}
%    \end{macrocode}
%Try to keep \LaTeX\ from overwriting its information files:
%    \begin{macrocode}
\@fileswfalse
%    \end{macrocode}
%Let the output routine throw everything gathered regularly away.
%Start with all float boxes, continue with output box, pack everything
%afloat from \cmd{\@currlist} back into \cmd{\@freelist}.
%    \begin{macrocode}
\output{\def\@elt#1{\global\setbox#1=\box\voidb@x}%
  \@currlist
  \@elt{255}%
  \let\@elt\relax
  \xdef\@freelist{\@currlist\@freelist}%
  \global\let\@currlist\@empty}
%</active>
%</style>
%    \end{macrocode}
%
% \section{Various drivers}
% The installer, in case it is missing.  If it is to be used via
% |make|, we don't specify an installation path, since
%\begin{quote}
%|make install|
%\end{quote}
% is supposed to cater for the installation itself.
%    \begin{macrocode}
%<installer> \input docstrip
%<installer&make> \askforoverwritefalse
%<installer> \generate{
%<installer>    \file{preview.drv}{\from{preview.dtx}{driver}}
%<installer&!make>    \usedir{tex/latex/preview}
%<installer>    \file{preview.sty}{\from{preview.dtx}{style}
%<installer>                       \from{preview.dtx}{style,active}}
%<installer>    \file{prauctex.def}{\from{preview.dtx}{auctex}}
%<installer> }
%<installer> \endbatchfile
%    \end{macrocode}
% And here comes the documentation driver.
%    \begin{macrocode}
%<driver> \documentclass{ltxdoc}
%<driver> \begin{document}
%<driver> \DocInput{preview.dtx}
%<driver> \end{document}
%    \end{macrocode}
% \Finale{}
