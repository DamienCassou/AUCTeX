% -*- plain-tex -*- 
% $Id: ref-card.tex,v 1.1 1992-07-22 13:06:40 krab Exp $
% Reference Card for Auc TeX version 5.4 for Emacs Lisp
%**start of header
\newcount\columnsperpage

% This file can be printed with 1, 2, or 3 columns per page (see below).
% Specify how many you want here.  Nothing else needs to be changed.

\columnsperpage=2

% This file is intended to be processed by plain TeX (TeX82).
% compile-command: "tex auc-tex-ref.tex"

% The final reference card has six columns, three on each side.
% This file can be used to produce it in any of three ways:
% 1 column per page
%    produces six separate pages, each of which needs to be reduced to 80%.
%    This gives the best resolution.
% 2 columns per page
%    produces three already-reduced pages.
%    You will still need to cut and paste.
% 3 columns per page
%    produces two pages which must be printed sideways to make a
%    ready-to-use 8.5 x 11 inch reference card.
%    For this you need a dvi device driver that can print sideways.
% Which mode to use is controlled by setting \columnsperpage above.

%                   Author of Auc-TeX Reference Card
%
%		  		\ ! /
%			       -- O --
%			  __    / ! \   __
%			  \   /\     /\  /
%			   \ /  \===/  \/
%				/   \
%			       /  x  \
%			      /   x   \
%				   
%       Terrence Brannon, PO Box 5027, Bethlehem, PA 18015 , USA
%  internet: tb06@pl118f.cc.lehigh.edu  (215) 758-1720 (215) 758-2104

%
% Thanks to Stephen Gildea
% Paul Rubin, Bob Chassell, Len Tower, and Richard Mlynarik
% for creating the GNU Emacs Reference Card from which this was mutated

\def\versionnumber{5.4}
\def\year{1992}
\def\version{January \year\ v\versionnumber}

\def\shortcopyrightnotice{\vskip 1ex plus 2 fill
  \centerline{\small \copyright\ \year\ Free Software Foundation, Inc.
  Permissions on back.  v\versionnumber}}

\def\copyrightnotice{
\vskip 1ex plus 2 fill\begingroup\small
\centerline{Copyright \copyright\ \year\ Free Software Foundation, Inc.}
\centerline{designed by Terrence Brannon, \version}
\centerline{for GNU Emacs version 18 on Unix systems}

Permission is granted to make and distribute copies of
this card provided the copyright notice and this permission notice
are preserved on all copies.


\endgroup}

% make \bye not \outer so that the \def\bye in the \else clause below
% can be scanned without complaint.
\def\bye{\par\vfill\supereject\end}

\newdimen\intercolumnskip
\newbox\columna
\newbox\columnb

\def\ncolumns{\the\columnsperpage}

\message{[\ncolumns\space 
  column\if 1\ncolumns\else s\fi\space per page]}

\def\scaledmag#1{ scaled \magstep #1}

% This multi-way format was designed by Stephen Gildea
% October 1986.
\if 1\ncolumns
  \hsize 4in
  \vsize 10in
  \voffset -.7in
  \font\titlefont=\fontname\tenbf \scaledmag3
  \font\headingfont=\fontname\tenbf \scaledmag2
  \font\smallfont=\fontname\sevenrm
  \font\smallsy=\fontname\sevensy

  \footline{\hss\folio}
  \def\makefootline{\baselineskip10pt\hsize6.5in\line{\the\footline}}
\else
  \hsize 3.2in
  \vsize 7.95in
  \hoffset -.75in
  \voffset -.745in
  \font\titlefont=cmbx10 \scaledmag2
  \font\headingfont=cmbx10 \scaledmag1
  \font\smallfont=cmr6
  \font\smallsy=cmsy6
  \font\eightrm=cmr8
  \font\eightbf=cmbx8
  \font\eightit=cmti8
  \font\eighttt=cmtt8
  \font\eightsy=cmsy8
  \textfont0=\eightrm
  \textfont2=\eightsy
  \def\rm{\eightrm}
  \def\bf{\eightbf}
  \def\it{\eightit}
  \def\tt{\eighttt}
  \normalbaselineskip=.8\normalbaselineskip
  \normallineskip=.8\normallineskip
  \normallineskiplimit=.8\normallineskiplimit
  \normalbaselines\rm		%make definitions take effect

  \if 2\ncolumns
    \let\maxcolumn=b
    \footline{\hss\rm\folio\hss}
    \def\makefootline{\vskip 2in \hsize=6.86in\line{\the\footline}}
  \else \if 3\ncolumns
    \let\maxcolumn=c
    \nopagenumbers
  \else
    \errhelp{You must set \columnsperpage equal to 1, 2, or 3.}
    \errmessage{Illegal number of columns per page}
  \fi\fi

  \intercolumnskip=.46in
  \def\abc{a}
  \output={%
      % This next line is useful when designing the layout.
      %\immediate\write16{Column \folio\abc\space starts with \firstmark}
      \if \maxcolumn\abc \multicolumnformat \global\def\abc{a}
      \else\if a\abc
	\global\setbox\columna\columnbox \global\def\abc{b}
        %% in case we never use \columnb (two-column mode)
        \global\setbox\columnb\hbox to -\intercolumnskip{}
      \else
	\global\setbox\columnb\columnbox \global\def\abc{c}\fi\fi}
  \def\multicolumnformat{\shipout\vbox{\makeheadline
      \hbox{\box\columna\hskip\intercolumnskip
        \box\columnb\hskip\intercolumnskip\columnbox}
      \makefootline}\advancepageno}
  \def\columnbox{\leftline{\pagebody}}

  \def\bye{\par\vfill\supereject
    \if a\abc \else\null\vfill\eject\fi
    \if a\abc \else\null\vfill\eject\fi
    \end}  
\fi

% we won't be using math mode much, so redefine some of the characters
% we might want to talk about
\catcode`\^=12
\catcode`\_=12

\chardef\\=`\\
\chardef\{=`\{
\chardef\}=`\}

\hyphenation{mini-buf-fer}

\parindent 0pt
\parskip 1ex plus .5ex minus .5ex

\def\small{\smallfont\textfont2=\smallsy\baselineskip=.8\baselineskip}

\outer\def\newcolumn{\vfill\eject}

\outer\def\title#1{{\titlefont\centerline{#1}}\vskip 1ex plus .5ex}

\outer\def\section#1{\par\filbreak
  \vskip 3ex plus 2ex minus 2ex {\headingfont #1}\mark{#1}%
  \vskip 2ex plus 1ex minus 1.5ex}

\newdimen\keyindent

\def\beginindentedkeys{\keyindent=1em}
\def\endindentedkeys{\keyindent=0em}
\endindentedkeys

\def\paralign{\vskip\parskip\halign}

\def\<#1>{$\langle${\rm #1}$\rangle$}

\def\kbd#1{{\tt#1}\null}	%\null so not an abbrev even if period follows

\def\beginexample{\par\leavevmode\begingroup
  \obeylines\obeyspaces\parskip0pt\tt}
{\obeyspaces\global\let =\ }
\def\endexample{\endgroup}

\def\key#1#2{\leavevmode\hbox to \hsize{\vtop
  {\hsize=.75\hsize\rightskip=1em
  \hskip\keyindent\relax#1}\kbd{#2}\hfil}}

\newbox\metaxbox
\setbox\metaxbox\hbox{\kbd{M-x }}
\newdimen\metaxwidth
\metaxwidth=\wd\metaxbox

\def\metax#1#2{\leavevmode\hbox to \hsize{\hbox to .75\hsize
  {\hskip\keyindent\relax#1\hfil}%
  \hskip -\metaxwidth minus 1fil
  \kbd{#2}\hfil}}

\def\threecol#1#2#3{\hskip\keyindent\relax#1\hfil&\kbd{#2}\quad
  &\kbd{#3}\quad\cr}

%**end of header


\title{AUC La\TeX\ Reference Card}

\centerline{(for version 5.4)}

\section{Conventions Used}

\key{Carriage Return}{RET}
\key{Tabular}{TAB}
\key{Linefeed}{LFD}

Mode variables: Refer to `\kbd{tex-site.el}' in the AUC-\TeX\
Distributions for customization

Entering LaTeX mode calls the value of text-mode-hook,
then the value of TeX-mode-hook, and then the value
of LaTeX-mode-hook.


\section{Shell Interaction}

\key{Print the DVI file}{C-c !}
\key{Run \TeX/LaTeX on buffer}{C-c C-a}
\key{Run \TeX/LaTeX on region}{C-c C-d}
\key{Preview dvi file}{C-c C-p}
\key{Next error in \TeX/LaTeX session}{C-c C-n}
\key{Run Bib\TeX\ on buffer}{C-c @}
\key{Run makeindex on buffer}{C-c \#}
\key{Kill job}{C-c C-k}
\key{Recenter output buffer}{C-c C-l}
\key{Toggle Debug Boxes}{C-c C-w}
\key{Home Buffer}{C-c C-h}

\section{Automatic Command Insertion}

\key{Insert Brace}{M-\{}
\key{Insert {\bf bold} syntax}{C-c C-b}
\key{Insert {\it italics\/}syntax}{C-c C-i}
\key{Insert {\rm roman} syntax}{C-c C-r}
\key{Insert emphasized syntax}{C-c C-e}
\key{Insert {\tt typewriter} syntax}{C-c C-t}
\key{Insert {\sl slanted\/} syntax}{C-c C-s}
\key{Insert small caps syntax}{C-c C-y}
\key{Insert Section/Chapter/etc syntax}{C-c C-x}
\key{Insert \kbd{\\begin\{\} ... \\end\{\}} environment}{C-c C-c}
\key{Insert \TeX\ macro \kbd{\\\{\}} }{C-c RET}
\key{\TeX command on region \kbd{\{\\ \}}}{C-c C-o}

\section{Automatic Commenting}

\key{Comment out a region}{C-c ;}
\key{Comment out a paragraph}{C-c '}
\key{Uncomment a region}{C-c :}
\key{Uncomment}{C-c "}
\key{Insert item}{M-RET}

\section{Marking and Formatting}

\key{Format a paragraph}{M-q}
\key{Format a region}{M-g}
\key{Mark a section}{M-C-x}
\key{Format a section}{M-s}
\key{Mark an Environment}{M-C-e}
\key{Close off an Environment}{C-c C-f}
\key{Format an Environment}{M-C-q}

\section{Miscellany}

\key{Complete Symbol}{M-TAB}
\key{Up-list}{M-\}}
\key{Terminate Paragraph}{RET}
\key{Smart ``Quote'' Insert}{"}
\key{LaTeX-indent-line}{TAB}
\key{Reindent, then newline and indent}{LFD}
\key{Terminate paragraph}{C-c LFD}

Type \kbd{TeX-validate-buffer} to insure that the \$ and braces
match up within paragraphs.

\copyrightnotice

\bye

% End:
